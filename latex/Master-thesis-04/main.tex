% !TeX encoding = UTF-8
% !TeX program = pdflatex
% !TeX spellcheck = en_US
\documentclass[binding=0.6cm,LaM]{sapthesis}
\usepackage{microtype}
\usepackage[english]{babel}
\usepackage[utf8]{inputenc}

%\usepackage{setspace}
%\singlespacing
%\linespread{0.9}
\usepackage[backend=bibtex,citestyle=authoryear,bibstyle=apa]{biblatex}
\bibliography{bibliography}

%%%% ZAVVE: to add parenthesis to \cite{} command
\newcommand{\mycite}[1]{(\cite{#1})}

\usepackage{hyperref}
\hypersetup{
    %hyperfootnotes=true,			
    %bookmarks=true,			
    %colorlinks=true,
    %linkcolor=red,
    %linktoc=page,
    %anchorcolor=black,
    %citecolor=red,
    %urlcolor=blue,
    pdftitle={Parametrized CF Explanations in Graph Neural Networks},
    pdfauthor={Giammarco D'Alessandro},
    pdfkeywords={thesis, sapienza, roma, university}
 }


\title{Parametrized Counterfactual Explanations for Node Classification in Graph Neural Networks}
\author{Giammarco D'Alessandro}
\IDnumber{1753102}
\course{Engineering in Computer Science - Ingegneria Informatica}
\courseorganizer{Facoltà di Ingengneria dell'Informazione, Informatica e Statistica}
\AcademicYear{2022/2023}
\advisor{Prof. Fabrizio Silvestri}
\coadvisor{Prof. Simone Scardapane}
%\examdate{... November 2023}
\authoremail{dalessandro.1753102@studenti.uniroma1.it}
\copyyear{2023}
%\thesistype{Master thesis}

\begin{document}

\frontmatter
\maketitle
\dedication{Dedicato a\\ Donald Knuth}

\begin{abstract}
Given the increasing popularity of Graph Neural Networks (GNNs) in real-world applications such as computational biology, natural language processing (NLP), and computer security, etc...; and the \textit{black-box} nature of such models, several methods have been developed for explaining their predictions. One recent approach to address this problem is counterfactual reasoning, where the goal of an explainer algorithm is to induce a change in the GNN prediction by minimal perturbation of the input structure. 

%Existing methods for interpreting predictions from GNNs have primarily focused on generating subgraphs that are especially relevant for a particular prediction. However, such methods are not counterfactual (CF) in nature: given a prediction, we want to understand how the prediction can be changed in order to achieve an alternative outcome. In this work, we propose a method for generating CF explanations for GNNs: the minimal perturbation to the input (graph) data such that the prediction changes. Using only edge deletions, we find that our method, CF-GNNExplainer, can generate CF explanations for the majority of instances across three widely used datasets for GNN explanations, while removing less than 3 edges on average, with at least 94\% accuracy. This indicates that CF-GNNExplainer primarily removes edges that are crucial for the original predictions, resulting in minimal CF explanations.

%Graph neural networks (GNNs) find applications in various domains such as computational biology, natural language processing, and computer security. Owing to their popularity, there is an increasing need to explain GNN predictions since GNNs are black-box machine learning models. One way to address this is counterfactual reasoning where the objective is to change the GNN prediction by minimal changes in the input graph. Existing methods for counterfactual explanation of GNNs are limited to instance-specific local reasoning. This approach has two major limitations of not being able to offer global recourse policies and overloading human cognitive ability with too much information. In this work, we study the global explainability of GNNs through global counterfactual reasoning. Specifically, we want to find a small set of representative counterfactual graphs that explains all input graphs. Towards this goal, we propose GCFExplainer, a novel algorithm powered by vertexreinforced random walks on an edit map of graphs with a greedy summary. Extensive experiments on real graph datasets show that the global explanation from GCFExplainer provides important high-level insights of the model behavior and achieves a 46.9% gain in recourse coverage and a 9.5% reduction in recourse cost
\end{abstract}

\tableofcontents
\mainmatter

%%%%%%%%%%%%%%%%%%%%%%%%% CHAP.1 : Introduction 
\chapter{Introduction}
\label{chap:1} 
In recent years, Graph Neural Networks (GNNs) have emerged as a powerful tool for analyzing complex relational data represented in the form of graphs. Graphs are pervasive in various domains, including social networks, biological systems, transportation networks, and recommendation systems. The ability of GNNs to model and understand these graph structures has led to significant advancements in tasks such as node classification, link prediction, and community detection.

Node classification, a fundamental task in graph analysis, involves predicting labels or categories for nodes in a graph based on their structural and attribute information. Despite the remarkable success of GNNs in node classification, interpreting their decisions remains a challenge. Understanding why a GNN predicts a specific label for a node is crucial for building trust, improving model robustness, and gaining insights into the underlying data.

One promising approach to address the interpretability challenge in GNNs is the integration of counterfactual explanations. Counterfactual explanations provide meaningful insights into model predictions by identifying what changes in the input features would lead to a different prediction. In the context of node classification, counterfactual explanations can elucidate the necessary alterations to the node's attributes or connections that would result in a different predicted label.

This thesis focuses on the incorporation of counterfactual explanations to enhance interpretability in GNNs for node classification. We explore how these explanations can shed light on the decision-making process of GNNs and provide actionable insights for stakeholders. The objective is to bridge the gap between predictive accuracy and model interpretability, facilitating the deployment of GNNs in real-world applications where transparency and trustworthiness are paramount.

In this introductory chapter, we present an overview of the research problem, define the research objectives, outline the scope and contributions of this thesis, and provide a brief outline of the subsequent chapters. Additionally, we discuss the significance of counterfactual explanations in the context of GNNs and node classification, setting the stage for the subsequent chapters that delve deeper into this research domain.


%%%%%%%%%%%%%%%%%%%%%%%%% CHAP.2 : Introduction
\chapter{Background}
\label{chap:2}
\section{Graph Neural Networks}
\label{sec:gnns}
Graph Neural Networks (GNNs, \cite{gnnModel2009}) represent a revolutionary paradigm in the field of machine learning, particularly suited for analyzing and extracting insights from complex structured data, such as graphs. Traditional neural networks excel at processing signals like images, speech, videos or sequences in which there is an underlying Euclidean structure, but they struggle with non-Euclidean data, like graphs. Such kinds of data implies that there are no such familiar properties as global parameterization, common system of coordinates, vector space structure, or shift-invariance. Consequently, basic operations like convolution that are taken for granted in the Euclidean case are even not well defined on non-Euclidean domains \mycite{Bronstein_2017}. 

Graph Neural Networks, on the other hand, have emerged as a powerful tool for handling these intricate data structures by incorporating graph-related properties. Recent developments have increased their capabilities and expressive power. We are starting to see practical applications in a really wide spectrum of different areas such as drug discovery \mycite{doi:10.1021/acs.jmedchem.9b00959}, physics simulations \mycite{sanchezgonzalez2020learning}, fake news detection \mycite{monti2019fake}, recommendation systems \mycite{eksombatchai2017pixie} and many others \mycite{hamilton2020graph}.

In essence, GNNs are a class of deep learning models designed to operate on graphs, capturing both the node-level and the edge-level information within a graph. They are engineered to learn meaningful representations of nodes and edges by leveraging the graph's topology and associated features. These representations can be exploited in performing various tasks, such as node classification, link prediction, community detection, and graph generation. More generally GNNs have shown remarkable performance in domains where relationships and interactions between entities are crucial for understanding system dynamics and making informed predictions.

This introduction will delve deeper into the foundations, architectures, and applications of Graph Neural Networks, providing a comprehensive understanding of this cutting-edge approach to processing complex relational data. First it will go through some basics of graph theory and it will analyze how these concepts can be used to model real world scenarios, and how neural networks  can exploit the information encoded in such models. Last it will added a brief presentation of different types of GNNs and their peculiarities.

\subsection{what are graphs, graphs in real world}
Graphs, in the realm of mathematics and computer science, are versatile data structures used to model relationships between entities. A graph G can be defined as the couple:
\begin{equation}
    G = <V,E>
\end{equation}
where $V = \{v_1,v_2,...,v_n\}$ is the set of \textit{vertices} or \textit{nodes} (the entities), and $E = \{(v_i,v_j) | v_i,v_j \in V; i,j = 1,...,n\}$ is the set of \textit{edges}, also called links, modeling the relationship between two entities.
A basic distinction is made between \textit{undirected} graphs, where edges link two vertices symmetrically, i.e. there is non difference between the edges $(v_i,v_j)$ and $(v_j,v_i)$; and directed graphs, where edges link two vertices asymmetrically. Graphs are one of the principal objects of study in discrete mathematics. 
These objects, often referred to as nodes or vertices, are the fundamental building blocks of the graph. Meanwhile, the connections between these nodes, called edges, depict the relationships or interactions between the objects they represent \mycite{cormen2022introduction}.

This graphical representation allows us to visualize complex relationships and structures, making graphs a powerful tool for a wide array of applications. Graphs provide a flexible framework to model various real-world scenarios such as social networks, transportation systems, molecular structures, communication networks, and more. The simplicity of the graph concept, along with its broad applicability, has made it a cornerstone in diverse fields including computer science, operations research, biology, social sciences, and linguistics.

\subsection{Message passing, GNN tasks}
\subsection{GNN tasks}
\subsection{different types of GNNs}

\section{Explainability - XAI}
\label{sec:xai}
Intro base sul problema della explainability/interpretability e sulla sua necessità
\subsection{Explainability vs Interpretability}
\subsection{Explainability in GNNs}
Lallero \mycite{ying2019gnnexplainer}
\subsection{Counterfactual reasoning}


%%%%%%%%%%%%%%%%%%%%%%%%% CHAP.3 : My method
\chapter{Parametrized Counterfactual Explainations for Node Classification in Graph Neural Networks}
\label{chap:3}
\section{Related works - Global CF}
\label{sec:gcf}
\section{Problem formal definition}
\label{sec:form-def}
\section{Proposed Approach}
\label{sec:cfpg}
\subsection{Discrete latent structures}
Lallero \mycite{niculae2023discrete}


%%%%%%%%%%%%%%%%%%%%%%%%% CHAP.4 : Experiments
\chapter{Experimental results}
\label{chap:4}

\section{Synthetic datasets}
\label{sec:syns}
Property of the European Southern Observatory...

\section{Runs setup}
\label{sec:moons}
The \textit{Multi-Object Optical and Near-infrared Spectrograph} is a future generation MOS instrument for the VLT. 


%%%%%%%%%%%%%%%%%%%%%%%%% CHAP.5 : Conclusions
\chapter{Conclusions}
\label{chap:5} 
The grasping power of the mirror..





\backmatter
\cleardoublepage % blank page after each chapter

%%%%%%%%%%%%%%%%%%%%%%%%% bibliography
\phantomsection % Give this command only if hyperref is loaded
\addcontentsline{toc}{chapter}{\bibname}
% Here put the code for the bibliography. You can use BibTeX or
% the BibLaTeX package or the simple environment thebibliography.
\printbibheading
\printbibliography[type=article,heading=subbibliography,title={Articles}]
\printbibliography[type=inproceedings,heading=subbibliography,title={Inproceedings}]
\printbibliography[type=misc,heading=subbibliography,title={Miscellaneous}]

\end{document}